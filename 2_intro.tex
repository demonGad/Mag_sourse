\chapter*{Введение}
\addcontentsline{toc}{chapter}{Введение}
\label{ch:intro}

%Актуальность -- Проблематика -- Цель

%Сердечно сосудистые заболевания попрежнему остаются основной причиной смертнос
Cердечно-сосудистые заболевания по-прежнему остаются ведущей причиной смертности взрослого населения в развитых странах \cite{1}, что обуславливает необходимость развития математических моделей кровотока как для расчета индивидуальных рисков, так и для поддержки клинических решений. Особую актуальность данное направление приобретает в сфере инвазивных процедур, таких как стентирование коронарных артерий, где расчет фракционного резерва кровотока (ФРК) критически важен для определения необходимости хирургического вмешательства.

Однако существующие одномерные модели кровотока сталкиваются с рядом ограничений, которые препятствуют их широкому введению в клиническую практику. Во-первых, данные модели требуют значительных вычислительных ресурсов из-за сложности методов, решающих уравнения, которые описывают пульсирующий поток крови. Также важна  оперативность результатов, что особенно актуально для таких задач, как интраоперационный мониторинг или экстренная диагностика. Во-вторых, многие модели зависят от параметров, которые невозможно получить неинвазивными методами. Например, данные о локальном давлении, жёсткости стенок сосудов или периферическом сопротивлении часто требуют проведения инвазивных процедур, таких как катетеризация, что сопряжено с риском для пациента и увеличивает стоимость обследования. Кроме того, такие измерения могут быть недоступны в рутинной практике, что ограничивает применимость моделей в стандартных клинических сценариях.

Третья проблема заключается в сложности персонализации моделей под индивидуальные анатомические и физиологические особенности пациентов. Необходимость ручной настройки параметров увеличивают время подготовки модели и риск ошибок, особенно при отсутствии полного набора диагностических данных. Это делает процесс трудоёмким и малопригодным для массового использования, где ключевыми факторами остаются простота и скорость интерпретации результатов.

%Windkessel == эластичного (упругого) резервуара

Для преодоления указанных ограничений в данной работе предлагается подход, объединяющий одномерную модель коронарного кровотока с моделью эластичного (упругого) резервуара, которая заменяет ресурсоемкие вычисления в одномерной виртуальной аорте на компактную 0D модель. При данном подходе снижается количество необходимых начальных параметров при сохранении значимости результатов. Новизна работы заключается в разработке адаптивных граничных условий для одномерной модели кровотока, которые позволят облегчить проведение персонализации модели под конкретного пациента и ускорить вычисления. 

В рамках исследования предлагается решение следующих задач:
\begin{enumerate}
	\item Изучить текущие модели кровотока и использующиеся граничные условия и их параметры.
	\item Разработать персонализированные граничные условия.
	\item Разработать механизм адаптации модели под данные конкретного пациента.
	\item Проверить решение на имеющейся пациентной базе.
	\item Сравнить результаты работы моделей с виртуальной аортой и без.
\end{enumerate}


%провести литературный обзор существующих моделей коронарного кровотока и сравнительный анализ одномерной модели с граничными условиями в виде виртуальной аорты и в виде адаптивной модели эластичного резервуара. Для валидации подхода будут использованы как синтетические данные, позволяющие сравнить результаты работы модели с результатами  предыдущей модели, так и клинические данные пациентов.
% 

% цель работы - четко и  задачи для достижения цели

%\begin{enumerate}
%	\item Изучить текущие модели кротока и испольщующиеся граничные условия и их параметры.
%	\item Разработать персонализированные граничные условия с учетом опыта коллег.
%	\item Проверить решение на имеющейся пациентной базе.
%\end{enumerate}


% разработать мат модель стыковки аорты с коронарными артериями
% По мат можели реализовать чиссленный метод для расчета уравнений в узле аорты
% интегрировать модель распределения потоков в аорте в можель одномерного кровотока
% предложить механизм адаптации аорты для персонадизации модеди
% валидация на данных реальзых пациентов





%Персонализированные модели играют важную роль в биомедицинских задачах, так как они учитывают индивидуальные особенности каждого пациента. Такие модели позволяют формализовать и связать ключевые патофизиологические процессы, оценивать важные для диагностики заболевания параметры и прогнозировать результаты терапевтических или хирургических вмешательств. Однако для успешного использования моделей в клинической практике необходимо обеспечить их валидацию, автоматизацию технологической цепочки от обработки входных данных до получения результата, а также высокую скорость расчетов \cite{василевский2022персонализация}. 

%Для автоматизации такой технологической цепочки необходимо разработать систему поддержки принятия врачебных решений (СППВР). Задача разработки СППВР включает в себя не только разработку математической модели, но и создание удобного интерфейса для взаимодействия с врачами, что особенно важно для эффективного применения в клинической практике \cite{shaalan2020visualization}.

%Для успешного внедрения СППВР в клиническую практику необходимо, чтобы система удовлетворяла следующим техническим критериям: масштабируемость, развертываемость, поддерживаемость, сопровождаемость, совместимость внутренних модулей. Эти критерии распространяются и на внутренние модули системы, в том числе на интерфейс. 

%Помимо технических критериев, самым важным является практическая значимость для врачей: разрабатываемая система должна уметь эффективно решать актуальную проблему врачей. 

%В рамках данной работы было предложено проанализировать проблемы старого веб-интерфейса СППВР для операции Озаки, изучить современные подходы к созданию веб-интерфейса и реализовать новую кодовую базу на основе этих фреймворков или доработать существующую.

%Цель данной работы --- разработка пользовательского веб-интерфейса в системе поддержки принятия врачебных решений для операции Озаки с использованием современных фреймворков.

%Список задач для достижения цели:
%\begin{enumerate}
   % \item Анализ предметной области: Изучить специфику операции Озаки и определить функциональные и нефункциональные требования к системе.
   % \item Анализ проблем предыдущего решения и современных подходов к созданию клиентских веб-приложений: Оценить старый веб-интерфейс, выявить его недостатки и рассмотреть современные подходы к разработке интерфейсов.
   % \item Реализация веб-интерфейса: Описать и реализовать новую кодовую базу с использованием современных фреймворков, описать интеграцию библиотек для решения возникающих задач, обеспечить интеграцию всех компонентов системы.
%\end{enumerate}

\endinput