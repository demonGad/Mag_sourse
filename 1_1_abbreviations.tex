\chapter*{Сокращения, обозначения, термины 
и определения}

В настоящей работе применяют следующие сокращения и обозначения:
\begin{itemize}

\item ФРК --- Фракционный резерв кровотока;

\item ССЗ --- Сердечно сосудистые заболевания;

\item 0D/1D --- Zero-dimensional/One-dimensional (Нольмерный/Одномерный).

\iffalse
\item AVD --- Aortic Valve Disease (Аортальный порок сердца);

\item AV-Rec --- Aortic Valve Reconstruction (Реконструкция аортального клапана);

\item CSS --- Cascading Style Sheets (Каскадные таблицы стилей);

\item DICOM --- Digital Imaging and Communications in Medicine (Цифровая обработка и передача медицинских изображений);

\item DOM --- Document Object Model (Объектная модель документа);

\item GCC --- Google Closure Compiler (Компилятор Google Closure);

\item GL --- Graphics Library (Графическая библиотека);

\item HTML --- HyperText Markup Language (Язык разметки гипертекста);

\item HTTPS --- HyperText Transfer Protocol Secure (Безопасный протокол передачи гипертекста);

\item JS --- JavaScript (Язык программирования JavaScript);

\item JSX --- JavaScript XML (Расширение синтаксиса JavaScript для использования XML/HTML);

\item КТ --- Компьютерная томография;

\item MVP --- Minimum Viable Product (Минимально жизнеспособный продукт);

\item СППВР --- Система поддержки принятия врачебных решений;

\item UI --- User Interface (Пользовательский интерфейс);

\item XML --- eXtensible Markup Language (Расширяемый язык разметки);

\item шейдеры —-- это программы, предназначенные для выполнения на графическом процессоре, которые определяют аспекты визуализации 3D-графики, включая цвета, освещение и тени;

\item 2D/3D --- Two-dimensional/Three-dimensional (Двумерный/Трехмерный).
\fi
\end{itemize}