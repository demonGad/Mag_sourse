\chapter*{Заключение}

Целью данной работы была разработка пользовательского веб-интерфейса в системе поддержки принятия врачебных решений для операции Озаки с использованием современных фреймворков. В ходе выполнения работы цель была достигнута. Был создан веб-интерфейс, реализующий следующие ключевые функции:

\begin{itemize}
    \item загрузка данных пациента и их просмотр: пользователи могут легко загружать КТ данные пациент в формате DICOM, а также просматривать их, используя основные функции такие как панорамирование, зумирование и вращение изображений.
    \item сегментация и визуализация аорты: обеспечена интеграция с серверной частью для выполнения задач по сегментации аорты и наглядного отображения результатов сегментации.
\end{itemize}

Разработанный веб-интерфейс основывается на современных инструментах, которые были выбраны путем анализа проблем старой кодовой базы и современных подходов к созданию веб-интерфейсов. Интерфейс реализован на языке TypeScript, использует сборщик Vite и фреймворк React. Для работы с 3D-графикой и медицинскими изображениями применяются библиотеки Three.js и Ami.js. Он соответствует современным требованиям и критериям, таким как масштабируемость, развертываемость, поддерживаемость и сопровождаемость.

Работа над проектом будет продолжена в рамках аспирантуры. Вторая часть разработки включает проведение линий пришивания створок клапана аорты, что станет следующим этапом совершенствования системы.

Код проекта доступен на github https://github.com/Micro-ice-ice/v5-react.

